%%%%%%%%%%%%%%%%%%%%%%%%%%%%%%%%%%%%%%%%%
% "ModernCV" Leadership Statement
% LaTeX Template
% Based on ModernCV Version 1.1 (9/12/12)
%
% Original author:
% Xavier Danaux (xdanaux@gmail.com)
%
% License:
% CC BY-NC-SA 3.0 (http://creativecommons.org/licenses/by-nc-sa/3.0/)
%
%%%%%%%%%%%%%%%%%%%%%%%%%%%%%%%%%%%%%%%%%

%---------------------------------------
%	PACKAGES AND OTHER DOCUMENT CONFIGURATIONS
%---------------------------------------

\documentclass[10pt,sans]{moderncv}

\moderncvstyle{classic}
\moderncvcolor{blue}


\usepackage{lipsum}
\usepackage{multicol}
\usepackage{graphicx}
\usepackage[margin=2.475cm]{geometry}

% Redefine section and subsection to add paragraph spacing
\let\originalsection\section
\let\originalsubsection\subsection
\renewcommand{\section}[1]{\originalsection{#1}\setlength{\parskip}{0.75em}\setlength{\parindent}{0pt}}
\renewcommand{\subsection}[1]{\originalsubsection{#1}\setlength{\parskip}{0.75em}\setlength{\parindent}{0pt}}

%---------------------------------------
%	NAME AND CONTACT INFORMATION SECTION
%---------------------------------------

\firstname{Nicholas}
\familyname{Hamilton}

\title{Leadership and Management Statement}
\address{3330 SW Knollbrook Ave.}{Corvallis, OR 97333}
\mobile{+1 720.551.1897}
\email{nicholas.hamilton@nrel.gov}
%\phone{(000) 111 1112}
%\fax{(000) 111 1113}
\email{nicholas.hamilton@nrel.gov}
%\homepage{staff.org.edu/~jsmith}{staff.org.edu/$\sim$jsmith} % The first argument is the url for the clickable link, the second argument is the url displayed in the FastFarm - this allows special characters to be displayed such as the tilde in this example
%\extrainfo{additional information}

%\photo[60pt][0.4pt]{mugshot.jpg} % The first bracket is the picture height, the second is the thickness of the frame around the picture (0pt for no frame)

%\quote{"A witty and playful quotation" - John Smith}

%---------------------------------------

\begin{document}

\makecvtitle
\vspace{-1cm}
\section{Leadership and Management of Multidisciplinary Teams}

PMEC's mission to advance marine energy from research to deployment while serving as trusted convener for Oregon's offshore energy future requires leadership that protects faculty autonomy, distributes decision-making authority, and translates \$240M+ infrastructure into sustained federal funding and coastal economic development. My approach centers on three priorities: building challenge-driven teams where coordination amplifies individual faculty impact, positioning PMEC strategically within the evolving federal research ecosystem, and maintaining an active research program that complements existing faculty expertise. This approach directly serves OSU's Prosperity Widely Shared strategic plan by expanding Oregon's research enterprise, strengthening team-based solutions-oriented work, and creating equitable outcomes through marine energy workforce development in underserved coastal regions. I am excited by the dual appointment structure, acting both as Director and tenure-track faculty, which would allow me to pursue sustained research leadership alongside PMEC's administrative mission.

\subsection{Portfolio Leadership and Multi-Project Coordination}

Multi-institutional research portfolios succeed when coordination amplifies individual investigator impact. Through managing concurrent projects spanning fundamental science, applied engineering, and infrastructure modernization, I've learned that effective coordination requires three elements: pre-deployment workshops and scientific scoping meeting establishing priorities before conflicts emerge, distributed leadership where working groups have decision authority, and regular synthesis enabling cross-institutional review.

AWAKEN (\$28M+) coordinated 12+ institutions across wind energy, atmospheric physics, and controls engineering through distributed working groups with decision authority, pre-deployment workshops establishing protocols, and scheduled synthesis sessions. More than 200 TB of reviewed data are now used internationally. This coordination model transfers directly to PMEC's multi-facility operations (PacWave, Hinsdale, WESRF, Gaulke) and multi-institutional scope (OSU-UW-UAF), where competing facility demands, heterogeneous sensor networks, and diverse stakeholder expectations create similar coordination challenges. Working with co-directors at UW and UAF, I would apply distributed leadership with clear decision rights, early coordination preventing bottlenecks, and regular synthesis maintaining strategic coherence.

My portfolio spans the ENDURA (\$7.8M+, multi-laboratory data utilization), FLOWMAS (\$19M+, exascale computational science), RAAW (\$8M+, industry validation frameworks) projects, infrastructure modernization at a world class R\&D facility (M5 Tower, ARM Mobile Facility deployments), and remote sensing innovation (acoustic tomography), hinging on my capability to balance competing priorities while maintaining federal relationships across DOE offices, NSF, NASA, and NOAA.

\subsection{Faculty Support and Center Coordination}

PMEC's facilities and multi-institutional structure create unprecedented opportunities for integrated systems research when coordination serves faculty ambitions. Effective leadership requires operating with transparency and integrity while cultivating conditions where individual researchers maintain scientific independence and pursue intellectually compelling questions. 
PMEC's multi-institutional structure (OSU-UW-UAF) and multi-facility coordination (PacWave, Hinsdale, WESRF, Gaulke Center) create opportunities to distribute expertise and share resources, but require preemptive coordination to prevent bottlenecks. My role as director would be reducing institutional friction to ensure clear facility access pathways, minimal coordination overhead, and decision-making authority remaining with research teams. Existing partnering workflows should be amplified where successful; where gaps create barriers, processes would be updated collaboratively.

\textbf{Facility access as faculty multiplier:} PacWave, Hinsdale, WESRF, and the Gaulke Center represent world-class capabilities whose value depends on transparent, equitable allocation balancing research quality with competing needs. Strategic allocation of facility time for proof-of-concept experiments between major campaigns enables junior faculty to develop fundable research programs strengthening PMEC's long-term capacity. Clear scheduling processes and early coordination with facility directors ensure sponsor deliverables, safety protocols, and institutional priorities are met.

\textbf{Distributed leadership and succession planning:} PMEC's institutional health requires distributing leadership across faculty, facility directors, and staff. Associate directors should have decision rights for specific domains (research strategy, facility coordination, industry partnerships, education). Transparent, minimal processes for facility allocation, proposals, and partnerships ensure continuity when key personnel transition. Administrative efficiency comes from clarity and consistency.

\textbf{Program area leadership:} PMEC's research portfolio spans marine and offshore energy conversion, power-at-sea applications, community engagement, multiphysics modeling, and environmental monitoring. Success requires identifying faculty leads for strategic program areas who coordinate proposals, represent PMEC at conferences, and mentor junior researchers. Rotating program leadership builds experience across the faculty, creates advancement pathways, and distributes decision authority. This way, each program area maintains autonomy over research priorities while contributing to PMEC's collective strategic positioning.

\textbf{Seed funding for proposal development:} PMEC's competitive advantage depends on rapid response to funding opportunities and reducing barriers for junior faculty. I would work with the Associate Dean for Research to develop a streamlined seed funding program supporting proposal development, proof-of-concept experiments, and preliminary data collection. This could include an open call requiring only 2-page concept papers demonstrating: (1) innovation not currently addressed, (2) credible path to external funding, (3) alignment with PMEC mission, and (4) facility requirements. Priority for junior faculty while remaining accessible to all for exploratory work reduces friction and accelerates proposal timelines.

\textbf{Transition:} Working with outgoing director Bryson Roberstson (OSU) and interim director Emma Cotter (PNNL), I would ensure continuity in PMEC operations and strategic direction during transition. Dr. Cotter's expertise in environmental monitoring technologies, machine learning for acoustic data, and marine energy-powered observing platforms complements my portfolio coordination and federal agency experience. Dr. Robertson's institutional knowledge and familiarity with the current trajectory of the center are of critical importance in ensuring minimal disruption to ongoing research activities. I see opportunities to build on their research and administrative leadership while bringing complementary capabilities in multi-institutional research portfolio development and upstream federal program coordination. Sustained collaboration with PNNL researchers will remain a strategic priority.

\subsection{Mentoring and Faculty Development}

PMEC's long-term success depends on enabling faculty to ask incisive questions and challenge assumptions, supporting students to become next-generation leaders, and building an organizational culture that creates opportunities through relationships. My approach balances technical skill development with career pathway guidance, connecting students to compelling research questions that align with fundable directions while building professional networks across national labs, industry, and academia.

Over my career, I have mentored 12 graduate students (MS and PhD), 4 postdoctoral researchers, 20+ undergraduate students, and 11 junior staff researchers, while serving on 6 thesis and dissertation committees. My three NREL Outstanding Mentor Awards (2018, 2020, 2023) reflect sustained success supporting diverse students from undergraduate interns to postdoctoral researchers. Several mentees now hold faculty positions, lead research programs at national laboratories, or occupy senior technical roles in industry. These trajectories reflect sustained engagement beyond graduation, including co-authoring papers, facilitating industry introductions, and advocating for their work in professional networks.

My mentoring philosophy centers on connecting researchers to intellectually compelling questions that align with fundable research directions while facilitating national lab partnerships at PNNL and NREL that provide sabbatical opportunities, student internships, postdoc placements, and access to computational and specialized experimental resources. At PMEC, I would prioritize creating structured mentoring for junior faculty through sabbatical facilitation, proposal development partnerships, and advocacy in tenure/promotion processes that recognize collaborative research contributions. This approach builds institutional bench strength through technical depth and mentoring relationships, positioning junior faculty as future PMEC leaders while sustaining sponsor relationships and stakeholder trust.

\textbf{Faculty recruiting:} PMEC's evolution requires strategic recruiting that complements existing expertise while addressing emerging priorities. I would work with school directors to identify candidates in areas where PMEC's facilities create competitive advantages including hybrid offshore energy systems, environmental monitoring, and community engagement research. Recruiting must emphasize diversity across gender, race, career stage, and disciplinary background while prioritizing scholars committed to collaborative, team-based research. Early-career hires benefit from PMEC's mentoring infrastructure and begin impactful research careers with access to state-of-the-art facilities and resources.

\subsection{Technology Transfer, Patents, and IP Development}

Marine energy's path to commercial viability requires translating research into deployable technologies, control algorithms, and design methodologies. PMEC should actively support faculty and students in developing intellectual property through patents, software records, and design tools that create licensing opportunities, industry partnerships, and revenue streams for Oregon State University.

Technology transfer creates multiple benefits: licensing royalties support OSU's research enterprise, patents strengthen grant proposals by demonstrating commercial relevance, startup formation creates career pathways for students and postdocs, and industry partnerships provide validation data and facility co-funding. For students, participating in patent development provides professional development beyond traditional academic training, creating competitive advantages for industry careers. This approach positions PMEC not just as research center but as innovation engine advancing Oregon's blue economy.

I will work with OSU's Office of Technology Transfer and Industry Partnerships to streamline IP disclosure processes, connect faculty with patent attorneys early in research trajectories, and identify commercialization pathways for PMEC-developed technologies. My experience developing DOE software records (Acoustic Tomography Processing Suite, Field Experiment Tool Arsenal) that have been adopted by industry partners demonstrates understanding of IP development in federally-funded research contexts, including navigating licensing terms, data rights, and technology transfer agreements.


\subsection{Teaching, Learning, and Workforce Development}

Teaching is central to PMEC's mission as a conduit for marine energy workforce development. I am prepared to teach core mechanical engineering courses (ME 331/560 Fluid Mechanics, ME 568 Turbulent Flow Dynamics, ME 552 Measurements in Fluid Mechanics and Heat Transfer), systems dynamics and control (ME 320/531), and building out the graduate seminar series (ME 599 Special Topics: Marine Energy Systems Engineering) with input from a broad cross section of renewable energy fields, including federal research from the national labs, academic partners, and industry developers. My academic appointment would be in the School of Mechanical, Industrial, and Manufacturing Engineering, reflecting my expertise in fluid mechanics, turbulence, and energy systems.

My teaching philosophy emphasizes a balance of lecture and project-based learning at the measurement-model interface where students develop physical intuition alongside analytical skills. PacWave and Hinsdale data provide ideal case studies where students design instrumentation, develop algorithms, and learn experimental design using real facility data. Beyond core curriculum, I can contribute to interdisciplinary courses in atmospheric science, data science, scientific machine learning, and instrumentation development, reflecting PMEC's multi-disciplinary research model. I would seek industry partners to serve as guest lecturers and project advisors, creating routes into the private sector, national laboratories, and regulatory agencies.

\subsection{Inclusive Excellence and Coastal Partnerships}

PMEC's inclusive excellence strategy integrates Tribal and coastal community partnerships for co-developed research priorities and environmental monitoring, workforce training with community colleges for marine technician roles (ROV pilots, electricians, environmental monitors), and retention support for underrepresented students through mentoring and funded internships. These partnerships advance OSU's Prosperity Widely Shared commitment to equity and Oregon's coastal economic development simultaneously. They build stakeholder relationships essential for PacWave operations and permitting while serving local economic development alongside federal sponsor interests.

Workforce development extends beyond traditional university pathways. I would build partnerships with Oregon Coast Community College and other regional institutions to develop technician training programs for marine energy careers: marine infrastructure programs using PacWave's facilities, marine electrical systems training for high-voltage subsea connections, and environmental monitoring technician programs supporting PMEC's stakeholder engagement with Tribal nations and fishing communities. These partnerships help build the skilled workforce PMEC needs for facility operations, directly advancing OSU's commitment to building prosperity across the state and equitable outcomes for learners from diverse and underrepresented backgrounds.

Beyond professional alignment, joining OSU's faculty holds personal significance. My family and I moved to Corvallis two years ago to build community and be closer to family—we're both Oregon natives. This transition from NREL to OSU represents commitment to Oregon's long-term research and economic development, not just career advancement. I'm invested in Corvallis as home and OSU as the institution where I can contribute most effectively to marine energy's future while remaining rooted in Oregon's coastal communities.

\section{Strategic Positioning Within the Federal Research Ecosystem}

PMEC is an OSU-led research center operating within a complex federal funding landscape. Success requires understanding agency priorities, positioning OSU as the preferred academic partner, and translating federal investments into sustained support for Oregon's coastal communities and research enterprise. My experience shaping research programs upstream of funding opportunity announcements and coordinating multi-institutional teams has been critical to translating between disciplinary communities with differing priorities.

\textbf{Early coordination and team formation:} Federal funding announcements typically have 60-90 day response windows for forming multidisciplinary teams and writing responsive proposals. Industry partnerships depend on demonstrating readiness and relevant expertise quickly. Proactive coordination requires maintaining awareness of faculty research interests and capabilities, understanding upcoming funding opportunities through program manager relationships, and facilitating introductions that reduce startup friction. When opportunities arise, faculty can find collaborators efficiently and access computational infrastructure, experimental facilities, and industry partnerships that amplify their research impact.

\subsection{Building with Federal Agencies and National Laboratories}

DOE's reorganization into the Office of Critical Materials and Energy Innovation (CMEI) shifts priorities toward grid resilience and blue economy integration. PMEC's multidisciplinary scope positions the center to address CMEI's integrated priorities: hybrid offshore systems, subsea autonomy for DOD applications, and environmental monitoring for NOAA's blue economy initiatives. Federal organizational transitions require adapting portfolio strategy to serve evolving sponsor missions. The fundamental science and core mission of PMEC remains unchanged even as the federal landscape shifts. Marine energy continues to play an important role in energy security, offshore infrastructure, and domestic supply chains.

Navigating this requires understanding not just what federal agencies fund, but how decisions are made. FOAs emerge from multi-year strategic planning processes involving program managers, office directors, and external advisory committees. Successful proposals align with these strategic priorities by articulating research questions in ways that serve sponsor missions while advancing fundamental understanding. My long-standing relationships with program managers across DOE (CMEI and Office of Science), NASA (JPL, EOS), NSF (Engineering, Geosciences), and NOAA provide channels to understand priorities early, connect PMEC faculty to opportunities before FOAs are released, and position OSU as a preferred partner when multi-institutional teams are needed.

National laboratories have long served as intermediaries between federal sponsors and academic research. They have mission-driven mandates, long-term funding stability, and facility capabilities not suited to academic institutions, but they depend on universities like OSU for fundamental research, a pipeline of student talent, and external validation of their technical approaches. Effective use of this symbiosis is critical to PMEC's success.

I have worked extensively with multi-lab teams on projects that span model development and validation, wind plant controls, and integration of hybrid energy systems into national power grids. These collaborations reflect years of co-authoring papers, co-developing proposals, sharing facility time, and embedding students for internships and postdoctoral positions. Partnerships across the national lab complex provide structured access for PMEC faculty to computational resources (NREL's HPC systems and test facilities), experimental facilities (PNNL's Marine Sciences Laboratory near PacWave), and specialized capabilities (e.g., Sandia's survivability modeling) while providing student career pathways through internships, postdoctoral positions, and staff scientist roles that demonstrate the value of PMEC training beyond academic tracks. The director's role is maintaining these relationships, understanding which lab capabilities align with faculty research interests, and facilitating introductions that reduce startup friction.

Legislative relationships complement programmatic partnerships. Oregon's congressional delegation and state legislators are critical partners for PMEC's success, requiring regular communication about research impact, facility capabilities, and economic development outcomes. I would work with OSU's Office of Government Relations to brief legislators on PMEC's contributions to Oregon's blue economy, coordinate facility tours demonstrating research impact, and provide technical expertise for marine energy policy development. These relationships ensure PMEC's value is understood beyond technical communities.

\subsection{Portfolio Architecture and Funding Opportunities}

PMEC's competitive advantages include PacWave as the only grid-connected, utility-scale open-ocean test site in the U.S., and multi-institutional scope spanning engineering, oceanography, environmental science, policy and economics. Co-location with OSU's offshore wind initiatives and Hatfield Marine Science Center enables hybrid marine energy system research and integrated environmental monitoring directly responsive to federal agency priorities (DOE, DOD, NOAA) and coastal stakeholders. These assets establish PMEC as the preferred academic partner for sponsors seeking to advance marine energy's role in grid resilience, offshore infrastructure, and blue economy development.

PMEC positions OSU as the state's essential partner for federal marine energy funding, channeling DOE, NSF, and DOD resources into Oregon's coastal economy. PacWave's \$240M+ infrastructure investment would not exist without OSU's leadership. Over the next decade, I envision PMEC evolving from device-focused testing toward integrated systems research spanning hybrid offshore platforms, grid forming and integration, global scale marine science, and economic benefit to coastal communities. This portfolio balances PMEC's distinctive and enduring capabilities with federal priorities while creating natural collaboration opportunities within OSU and with external partners. The goal is identifying themes where coordination amplifies individual impact and where PMEC's collective capabilities create competitive advantages for funding.

% Successful portfolio architecture requires identifying where outstanding scientific questions align with federal development priorities, then assembling multidisciplinary teams positioned to address both. My portfolio development approach balances fundamental research (exascale computational science through FLOWMAS), applied engineering (multi-laboratory field campaign coordination through ENDURA), and infrastructure modernization (measurement capabilities enabling future campaigns).

\subsection{Stakeholder Engagement and Partnership Management}

Effective federal program leadership requires engaging stakeholders without allowing any single perspective to dominate research design. This balance maintains trust across marine energy stakeholder groups, including regulatory agencies (BSEE and BOEM, environmental compliance, operational safety), DOE (grid reliability, deployment acceleration), WEC developers and industry partners (design standards, warranty costs), and coastal communities (multiple needs, uses, and benefits from future development).

This year I supported development of offshore wind survivability standards for BSEE and BOEM, separating scientific questions (failure modes, control strategies, load minimization) from policy questions (safety margins, inspection responsibilities, risk acceptance). Research informed regulatory standards, industry partners provided operational data, and regulatory agencies shaped inquiry scope. Maintaining research independence served all stakeholder missions.

Industry partnerships require similar balance. I would establish an industry advisory board meeting semi-annually to inform research priorities without controlling faculty agendas, create transparent processes for facility access and sponsored research, and develop partnership frameworks where industry provides validation data and co-funding while faculty maintain publication rights and intellectual independence.

The same principle applies to coastal communities: partnerships inform research questions while academic independence remains protected. The director's role is maintaining stakeholder relationships while ensuring research integrity and faculty independence. Co-developing research priorities with the Confederated Tribes of Siletz and Coos, partnering with fishing communities on environmental monitoring, and training coastal technicians for marine energy jobs directly advances OSU's commitment to prosperity widely shared.

\subsection{National Center Collaboration and Strategic Differentiation}

PMEC operates within a national marine energy ecosystem including the Atlantic Marine Energy Center (University of New Hampshire), Southeast National Marine Renewable Energy Center (Florida Atlantic University), and National Marine Renewable Energy Centers at Hawaii and Alaska. Strategic collaboration amplifies collective impact while clear differentiation prevents duplication and strengthens individual center value propositions. I already have a professional relationship with Dr. Martin Wosnik, the director of the Atlantic Marine Energy Center, through a shared academic lineage in turbulent fluid mechanics.

PMEC's unique assets with PacWave as the only grid-connected utility-scale open-ocean test site, Hinsdale's precision-controlled environment, and OSU's proximity to offshore wind lease areas create natural differentiation from Atlantic-focused centers. I would initiate coordination discussions with Atlantic Marine Energy Center, Hawaii, and Alaska centers within my first six months, focusing on methodology transfer, validation benchmark development, and joint advocacy for marine energy within federal agencies. Strategic collaboration on wave climate characterization, tropical wave energy applications, and remote community microgrid integration creates network effects benefiting all centers while preserving PMEC's differentiation through West Coast-specific priorities: offshore wind integration, seismic zone device survivability, and Pacific fisheries stakeholder engagement.

\section{Research Areas}

My research program complements existing PMEC faculty expertise while serving the center's strategic mission. Three interconnected areas address critical gaps:

\textbf{Wave Field State Estimation and Predictability Across Timescales:} Marine energy deployment and grid valuation require predictability across operational (minutes-hours), grid scheduling (hours-days), and survivability (storm events) timescales. State estimation under partial observability—inferring coupled wind-wave-device dynamics from sparse, heterogeneous measurements in energetic environments—remains a fundamental challenge limiting both device performance and facility operations.

Building on methods I developed for wind energy state estimation during AWAKEN and RAAW, I would develop physics-constrained data assimilation frameworks for wave field reconstruction using PacWave's distributed sensor network. This transforms operational data into predictive, uncertainty-aware environmental states actionable for device developers, operators, and grid partners. The work creates technical bridges to Prof. Hollinger's autonomous underwater vehicle research: improved hydrodynamic prediction reduces docking uncertainty in high sea states, while AUV sensors close the observational loop. Scientifically, this advances understanding of observability in spatially distributed nonlinear systems; practically, it enhances PacWave's unique capabilities as a national test facility providing validated environmental characterization unavailable elsewhere.

\textbf{Digital Twin Architectures for Real-Time Decision Support:} Marine energy technologies face tight coupling between design, environment, control, and lifetime performance. High-fidelity multiphysics models (Prof. Brian Johnson, UW) provide essential understanding but remain computationally intractable for real-time fault detection, control optimization, or operational decisions.

My research develops reduced-order digital twin architectures inheriting structure from high-fidelity models while enabling real-time applications. Validation follows a staged pipeline: controlled experiments at Hinsdale establish baseline behavior, long-duration PacWave deployments capture realistic forcing and degradation—directly aligned with PMEC's ALFA-LCP protocols (Profs. Robertson, Polagye). This enables model-predictive control for wave-to-wire optimization and fatigue management, creating alignment with OSU power electronics research (Profs. Brekken, Lomonaco) and Prof. Loeffler's (UAF) microgrid integration work for remote Alaskan communities. Collaboration with Prof. DuPont (OSU) on robust optimization and Prof. Brunton (UW) on physics-informed learning addresses compact representation of nonlinear fluid-structure-electromechanical coupling without sacrificing interpretability or operational relevance.

\textbf{Array-Scale Energy Transport and Environmental Interaction:} Array-scale deployment modifies wave climate, affecting device performance, environmental systems, and stakeholder acceptance. Understanding of energy transport and dissipation mechanisms under realistic conditions remains incomplete, creating both scientific gaps and barriers to responsible commercial-scale deployment.

I would pursue coordinated experimental-field research quantifying how array configurations modify near-field and far-field energy flux, spectral characteristics, and downstream wave climate. Hinsdale studies enable systematic investigation of spacing strategies and frequency-selective control; PacWave provides the only U.S. platform for utility-scale validation under realistic forcing. This interfaces directly with PMEC's environmental monitoring and community engagement priorities: quantitative wave climate modification metrics provide scientific foundation for Prof. Boudet's stakeholder research, connecting physical predictions to community concerns and regulatory requirements. Results inform monitoring protocols essential for social license and support PMEC's role as evidence-based convener—addressing regulatory science needs rather than fluid mechanics divorced from application context.

\section{Closing}

PMEC's infrastructure investments (PacWave, Hinsdale, WESRF) represent Oregon's commitment to marine energy as economic development and research capability. Success means leveraging that infrastructure for sustained federal funding, faculty career advancement, and coastal economic development. The directorship requires someone who understands how federal agencies make portfolio decisions, coordinates multi-institutional teams, and demonstrates how research creates value for Oregon's coastal communities.

My decade coordinating international field campaigns, building relationships with DOE program managers, and mentoring students from undergraduate interns to postdocs has prepared me for this role. Building on the foundation established by Drs. Robertson and Cotter at OSU, as well as Prof. Loeffler from UAF and Dr. Basset from UW APL, I'm positioned to advance PMEC through its next phase of strategic growth. Marine energy's integration into grid resilience, offshore wind, and blue economy priorities creates opportunities for faculty impact that didn't exist five years ago. PMEC stands at a decision point where strategic positioning within the federal research ecosystem can amplify individual faculty success and strengthen Oregon's leadership in marine energy innovation.

The dual appointment structure of leading PMEC while maintaining an active research program will enable me to model the collaborative, challenge-driven research culture the center requires. My family's decision to make Corvallis home reflects the same long-term commitment: I'm invested not just professionally but personally in Oregon's future and prosperity. Success will be measured through sustained federal funding growth, faculty career advancement, student placement in marine energy careers, and quantifiable economic impact in Oregon's coastal communities. I am committed to leading PMEC through this transformation.

\end{document}























