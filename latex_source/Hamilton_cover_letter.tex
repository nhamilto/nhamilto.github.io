%%%%%%%%%%%%%%%%%%%%%%%%%%%%%%%%%%%%%%%%%
% "ModernCV" CV and Cover Letter
% LaTeX Template
% Version 1.1 (9/12/12)
%
% This FastFarm has been downloaded from:
% http://www.LaTeXTemplates.com
%
% Original author:
% Xavier Danaux (xdanaux@gmail.com)
%
% License:
% CC BY-NC-SA 3.0 (http://creativecommons.org/licenses/by-nc-sa/3.0/)
%
% Important note:
% This FastFarm requires the moderncv.cls and .sty files to be in the same 
% directory as this .tex file. These files provide the resume style and themes 
% used for structuring the document.
%
%%%%%%%%%%%%%%%%%%%%%%%%%%%%%%%%%%%%%%%%%

%---------------------------------------
%	PACKAGES AND OTHER DOCUMENT CONFIGURATIONS
%---------------------------------------

\documentclass[10pt,sans]{moderncv} % Font sizes: 10, 11, or 12; paper sizes: a4paper, letterpaper, a5paper, legalpaper, executivepaper or landscape; font families: sans or roman

\moderncvstyle{classic} % CV theme - options include: 'casual' (default), 'classic', 'oldstyle' and 'banking'

\moderncvcolor{blue} % CV color - options include: 'blue' (default), 'orange', 'green', 'red', 'purple', 'grey' and 'black'

\usepackage{lipsum} % Used for inserting dummy 'Lorem ipsum' text into the FastFarm
\usepackage{multicol}
\usepackage{graphicx}
\usepackage[margin=2.475cm]{geometry} % Reduce document margins
%\setlength{\hintscolumnwidth}{3cm} % Uncomment to change the width of the dates column
%\setlength{\makecvtitlenamewidth}{10cm} % For the 'classic' style, uncomment to adjust the width of the space allocated to your name


%---------------------------------------
%	NAME AND CONTACT INFORMATION SECTION
%---------------------------------------

\firstname{Nicholas} % Your first name
\familyname{Hamilton} % Your last name

% All information in this block is optional, comment out any lines you don't need
\title{Curriculum Vitae}
\address{3330 SW Knollbrook Ave.}{Corvallis, OR 97333}
\mobile{+1 720.551.1897}
%\phone{(000) 111 1112}
%\fax{(000) 111 1113}
\email{nicholas.hamilton@nrel.gov}
%\homepage{staff.org.edu/~jsmith}{staff.org.edu/$\sim$jsmith} % The first argument is the url for the clickable link, the second argument is the url displayed in the FastFarm - this allows special characters to be displayed such as the tilde in this example
%\extrainfo{additional information}

%\photo[60pt][0.4pt]{mugshot.jpg} % The first bracket is the picture height, the second is the thickness of the frame around the picture (0pt for no frame)

%\quote{"A witty and playful quotation" - John Smith}


\begin{document}

%---------------------------------------
%	COVER LETTER
%---------------------------------------

% To remove the cover letter, comment out this entire block

\clearpage
%
\recipient{\vspace{-2cm} Search Committee }{
    Pacific Marine Energy Center Director Position  \\
    College of Engineering  \\
    Oregon State University
} % Letter recipient
\date{\today} % Letter date
\opening{Dear Members of the Search Committee} % Opening greeting

\closing{Best Regards,\\ \vspace{0cm}
\includegraphics[scale=1]{signature.png}\vspace{-1cm}} % Closing phrase
\enclosure[Attached]{CV, portfolio, Leadership and Management Statement, References} % List of enclosed documents

\makelettertitle % Print letter title

I am writing to apply for the combined position of Pacific Marine Energy Center Director and Associate Professor in the School of Mechanical, Industrial, and Manufacturing Engineering. PMEC stands at a transformative moment: PacWave operating as the nation's only grid-connected open-ocean test site, \$240M+ in testing infrastructure, and a multi-institutional consortium spanning Oregon State University, University of Washington, and University of Alaska Fairbanks. My experience directing \$65M+ research portfolios at the National Laboratory of the Rockies (formerly NREL), coordinating international field campaigns involving more than a dozen institutions, and maintaining strategic relationships with federal program managers across DOE, NSF, NASA, and NOAA positions me to lead PMEC through its next phase of growth.

My family and I moved to Corvallis two years ago, returning to Oregon where both my wife and I grew up. This decision reflected our commitment to building community and contributing to Oregon's future. Living here has fostered a real appreciation for the connections between OSU's research mission and the state's communities, and I'm encourage to see that emphasis in PMEC's work with Tribal nations, fishing communities, and marine energy workforce development. The opportunity to lead PMEC while contributing to OSU's campus community aligns perfectly with my professional goals and personal values around place-based research impact. OSU's Prosperity Widely Shared strategic plan resonates deeply with my commitment to translating research infrastructure into coastal economic development and equitable educational pathways.

My vision for PMEC centers on three strategic priorities: advancing PMEC as the catalyst for responsible offshore energy development along the Pacific coast, building distributed leadership that amplifies faculty impact through transparent facility access and strategic coordination, and translating PMEC's infrastructure into sustained federal funding by aligning marine energy with grid resilience, blue economy, and energy security priorities. Working with outgoing director Bryson Robertson and interim director Emma Cotter, I would ensure continuity in PMEC operations and strategic direction during transition. Dr. Robertson's institutional knowledge and familiarity with PMEC's current trajectory are critical for ensuring minimal disruption to ongoing research activities. Dr. Cotter's expertise in environmental monitoring and her established relationships across PMEC's consortium and stakeholder communities provide an excellent foundation. I see opportunities to build on their research and administrative leadership while bringing complementary capabilities in multi-institutional research portfolio development, federal agency coordination, and legislative engagement.

My leadership roll in the American WAKE experimeNt (AWAKEN, \$28M+) and Rotor Aerodynamics, Aeroelastics, and Wake (RAAW, \$8M+) projects demonstrates capability at PMEC's scale and complexity. These campaigns coordinated 12+ institutions across multiple disciplines through distributed working groups with decision authority, pre-deployment workshops establishing protocols, and scheduled synthesis sessions, resulting in 200+ TB of benchmark data now used internationally. The coordination model succeeded because it served scientific objectives while respecting institutional cultures and autonomy. This approach transfers directly to PMEC's OSU-UW-UAF consortium structure and multi-facility coordination challenges. My portfolio spanning fundamental research (FLOWMAS, \$19M+ exascale computational science), applied engineering (ENDURA, \$7.8M+ multi-laboratory data utilization), and infrastructure modernization demonstrates capability to balance competing priorities while maintaining federal relationships across multiple agencies.

PMEC's multi-facility coordination and diverse stakeholder landscape require transparent resource allocation and strategic engagement. My experience managing NREL's Flatirons Campus and industry partner facilities for AWAKEN/RAAW, combined with navigating IEC compliance, industry IP concerns, and BSEE/BOEM regulatory standards development, provides direct experience with this complexity. I would establish transparent facility allocation processes that prioritize research quality while enabling junior faculty to develop fundable programs. For all stakeholders, (e.g., Tribal nations, fishing communities, environmental groups, wave energy industry, Oregon Sea Grant, Pacific Ocean Energy Trust, regulators) I will ensure that input informs priorities without controlling conclusions, engaging communities in monitoring priorities, supporting developers with testing protocols, and advising regulators while protecting faculty research autonomy. For industry partners, I would create advisory structures and partnership frameworks where companies provide validation data and co-funding while faculty maintain publication rights and intellectual independence.

To support the center's endurance as a premier desination for marine science and energy R\&D, I would work with OSU's Office of Government Relations to engage Oregon's congressional delegation and state legislators. PMEC's value directing federal research dollars to coastal communities, workforce development, and Oregon's leadership in marine energy must be communicated effectively to policymakers. I am prepared to provide technical and strategic briefings, coordinate facility tours demonstrating research impact, and serve as a resource for marine energy policy development. These relationships ensure PMEC's contributions to Oregon's blue economy are understood beyond technical communities.

My research agenda expands PMEC's portfolio while leveraging existing collaborations. Wave field predictability and state estimation work using PacWave's instrumentation network supports Prof. Hollinger's AUV docking project and addresses grid integration requirements. Reduced-order modeling for coupled device-wave systems builds on Prof. Brian Johnson's (UW) multiphysics foundations while creating collaboration pathways with Profs. Brekken and Lomonaco (OSU) on control systems, Prof. Brunton (UW) on physics-informed machine learning, and Prof. DuPont (OSU) on optimization under uncertainty. Array-scale energy transport research addresses environmental monitoring and stakeholder engagement priorities while providing quantitative basis for responsible deployment. This agenda leverages PacWave's open-ocean deployment capabilities and Hinsdale's precision wave basin while strengthening technical connections across PMEC's three-university consortium.

Teaching and workforce development are central to PMEC's mission and my approach to faculty engagement. My sustained mentoring success (reflected in three NREL Outstanding Mentor Awards and mentees now holding faculty positions, leading laboratory programs, and occupying senior industry roles) demonstrates commitment to developing the next generation of marine energy researchers. At OSU, I am prepared to teach core mechanical engineering courses (ME 331/560 Fluid Mechanics, ME 568 Turbulent Flow Dynamics, ME 552 Measurements in Fluid Mechanics and Heat Transfer), systems dynamics and control (ME 320/531), and graduate seminars (ME 599 Special Topics: Marine Energy Systems Engineering). My teaching philosophy emphasizes project-based learning at the measurement-model interface, using PacWave and Hinsdale data for hands-on facility experience. Beyond traditional university pathways, I would build on partnerships with Oregon Coast Community College and regional institutions to develop technical training programs for marine careers in infrastructure operations, electrical systems, and environmental monitoring, directly advancing OSU's commitment to equitable outcomes and coastal workforce development.

My research productivity (1,200+ citations, h-index 21, i10-index 32) demonstrates sustained scholarly output alongside administrative leadership. As Associate Editor for the \emph{Journal of Renewable and Sustainable Energy} and Scientific Committee Lead for the 2026 North American Wind Energy Academy Symposium, I maintain engagement with international research communities. My work on wake dynamics, turbulence modeling, and coupled fluid-structure systems has generated datasets and validation benchmarks now used internationally, while my acoustic tomography development has advanced field research capabilities for atmospheric measurements.

The PMEC directorship represents the natural evolution of my career trajectory: applying proven coordination capabilities from large-scale field campaigns to marine energy's unique challenges while continuing active research using world-class facilities. Success means sustained federal funding growth, faculty career advancement through strategic coordination, student placement in marine energy careers, and measurable economic impact in Oregon's coastal communities. I am prepared to begin immediately upon appointment, working with co-Directors at UW and UAF and facility directors to execute PMEC's strategic plan, expand funding sources through diverse federal and industry partnerships, and position OSU as the nation's premier academic partner for marine energy development. I look forward to discussing how my experience can advance PMEC's mission and contribute to OSU's Prosperity Widely Shared strategic vision.

%\begin{figure}[h!]
%\includegraphics[scale = 1]{signature.png}
%\end{figure}

\makeletterclosing % Print letter signature

\end{document}























